
\documentclass[11pt, a4paper, sans]{moderncv}

\moderncvstyle [symbols]{classic}

\moderncvstyle{classic}
\moderncvcolor{blue}

\usepackage{geometry}
\usepackage[english]{babel}
\usepackage[utf8]{inputenc}
% \usepackage{hyperref}
\usepackage{fontawesome5}

\geometry{scale=0.8}
\setlength{\hintscolumnwidth}{4cm}
\setlength{\separatorcolumnwidth}{0.05\textwidth}

% https://tex.stackexchange.com/questions/310612/moderncv-date-year-on-top-month
\usepackage{datetime}
\def\dates[#1.#2-#3.#4]{\makebox[\hintscolumnwidth][c]{\yearabove{#1}{#2} -- \yearabove{#3}{#4}}}
\def\date[#1.#2]{\makebox[\hintscolumnwidth][c]{\yearabove{#1}{#2}}}
\newcommand{\yearabove}[2]{\parbox[t]{10mm}{\centering{#2\par\vspace{-2mm} \tiny{#1}}}}

% \makeatletter
% \renewcommand*{\cvitem}[2]{%
%   \cvline{%
%     \makebox[\hintscolumnwidth][c]{\strut\textcolor{color2}{#1}}%
%   }{#2}}
% \makeatother


\name{\textcolor{gray}{Christian}}{Döring}
\title{Graduate Student}
% \born{22.03.2001}
\email{doeringc2001@gmail.com}
\homepage{doeringc.de}
\social[github]{doeringchristian}
\social[orcid]{0009-0007-4763-8748}

\begin{document}

\makecvtitle

\section{Education}

\cventry{\dates[April.2023-.present]}
{M.Sc. Electrical and Computer Engineering}
{\newline Technical University of Munich}{}{}{}

\cventry{\dates[October.2019-March.2023]}
{B.Sc. Electrical and Computer Engineering}
{\newline Technical University of Munich}{}{}
{
	Thesis Title: Evaluation of Differentiable Inverse Rendering using Multi-View
	RGB Data
}

\cventry{\dates[September.2011-June.2019]}
{Abitur (A-Levels)}
{Gymnasium Bruckmühl}{}{}{}

\section{Publications}

\cvitem{\date[.2024]}{
	\textbf{Real-time Neural Rendering of Dynamic Light Fields}
	\newline
	Arno Coomans,
	Edoardo A. Dominici,
	Christian Döring,
	Joerg H. Mueller,
	Jozef Hladky,
	Markus Steinberger
	\newline
	\href{https://arnocoomans.be/eg2024/}{\faFile* Project} \hspace{1cm}
	\href{https://doi.org/10.1111/cgf.15014}{\faFilePdf \hspace{1pt} In Computer Graphics Formum (EG), 2024}
}

\section{Work Experience}

\cvitem{\dates[April.2025-.present]} {
	\textbf{Research Intern}
	\textit{NVIDIA}, Zurich
	\begin{itemize}
		\item Differentiable Rendering
		\item Development on Dr.Jit/Mitsuba3
	\end{itemize}
}
\cvitem{\dates[April.2024-April.2025]} {
	\textbf{Research Working Student} \textit{Huawei Technologies}, Munich
	\begin{itemize}
		\item Development on Dr.Jit/Mitsuba3
		\item Real-time Neural Rendering Research
	\end{itemize}
}
\cvitem{\dates[August.2023-February.2024]} {
	\textbf{Research Intern} \textit{Huawei Technologies}, Munich
	\begin{itemize}
		\item Real-time Neural Rendering Research
	\end{itemize}
}
\cvitem{\dates[July.2021-August.2021]} {
	\textbf{Embeded Systems Intern} \textit{Aurum GmbH}, Munich
	\begin{itemize}
		\item Developed NFC library for STM32 in C
	\end{itemize}
}
\cvitem{\date[July.2017]} {
	\textbf{Intern} \textit{Lauterbach GmbH}
}
\cvitem{\date[July.2017]} {
	\textbf{Intern} \textit{Electronic Theater Controls (ETC)}, Holzkirchen
}

\section{Side Projects}

\cvitem{}{
	\textbf{Hephaestus},
	Just In Time Compiler (JIT) for Vulkan, inspired by Dr.Jit. Implemented with own
	render graph solution. Includes cooperative matrix multiplication (KHR) and a
	port of tiny-cuda-nn in GLSL.
	\newline
	\href{https://github.com/doeringchristian/hephaestus-jit}{\faGit* Source}
}
\cvitem{}{
	\textbf{Vulkan Path Tracer},
	Path tracer written in Rust using the screen-13 library. It supports the Disney
	BSDF with Next Event Estimation.
	\newline
	\href{https://github.com/DoeringChristian/vulkan-rt}{\faGit* Source}
}
\cvitem{}{
	\textbf{Mitsuba3 Experiments},
	Implementation of forward and differentiable path tracing algorithms in
	Mitsuba3, such as
	\href{https://github.com/DoeringChristian/restirgi}{ReSTIR GI} and
	\href{https://github.com/mitsuba-renderer/mitsuba3/discussions/600}{Large
		Steps in Inverse Rendering}
}

\section{Skills}

\cvdoubleitem{Programming:}{
	\begin{itemize}
		\item \textbf{Rust}, C/C++
		\item \textbf{Vulkan}, CUDA
		\item \textbf{Python}, Lua
		\item \textbf{LaTeX}, Typst
	\end{itemize}
}
{Languages:}{
	\begin{itemize}
		\item \textbf{German} (native)
		\item English (fluent B2+/C1)
	\end{itemize}
}


\end{document}
