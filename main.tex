


\documentclass[11pt, a4paper, sans]{moderncv}

\usepackage[utf8]{inputenc}
\usepackage[scale = 0.8, top = 1.5cm, bottom = 1.5cm]{geometry}
\usepackage[export]{adjustbox}
\usepackage{ragged2e}
\usepackage{xcolor}
\usepackage{anyfontsize}
\usepackage[ngerman]{babel}
\usepackage[ddmmyyyy]{datetime}
\renewcommand{\dateseparator}{.}


\moderncvtheme[blue]{casual}
\sethintscolumnlength{4cm}

\title{Curriculum Vitae}
\firstname{Christian}
\familyname{Döring}
\mobile{+4915163484441}
\email{christian.doering@tum.de}

\begin{document}

\maketitle


\section{Education}
\cventry{Sept. 2011 - June 2019}{Abitur (A-Levels)}{Gymnasium Bruckmühl}{}{}{}

\cventry{Oct. 2019 - Mar 2023}{B.Sc., Electrical and Computer Engineering}
{\newline Technical University of Munich}
{\newline Thesis Title: Evaluation of Differentiable Inverse Rendering using Multi-View RGB Data}
{}{}

\cventry{Since Apr. 2023}{M.Sc., Electrical and Computer Engineering}
{\newline Technical University of Munich}
{}{}{}

\section{Internships}
\cventry{Aug 23 - Feb 24, 2023}{Intern, Neural Rendering Researcher}{Huawei Technologies}{Munich}{}{}{}
\cventry{July 26 - Aug 28, 2021}{Developer}{Aurum GmbH}
{Munich}{}
{Development of an RFID/NFC interface Device for writing to protectable memory of IoT sensors.}
\cvlistitem{NFC protocol standard e.g. iso14443}
\cvlistitem{OOP like programming in C99}

\cventry{July 10 - July 14, 2017}{Intern, client support}{Electronic Theater Controls (ETC)}{Holzkirchen}{}{}{}
\cventry{July 17 - July 21, 2017}{}{Lauterbach GmbH}{Höhenkirchen-Siegertsbrunn}{}{}{}

\section{Technical Experience}
% \cvline{Bachelor Thesis}{Evaluation of Differentiable Inverse Rendering using
% 	Multi-View RGB Data}
\cvline{AI controlled model car}{Implementation of a neural network framework in C++ for
	controlling a model car with a Raspberry Pi for a school project.
}
% \newline A model car was equipped with five ultrasonic sensors. Then the neural network
% was pre-trained using an evolutionary algorithm in a simulated environment. In
% the end this neural network was connected to the input sensors of the car as
% well as to the controls. Tests where conducted to evaluate the object avoidance
% capabilities of the vehicle.}
\cvline{Vulkan-rt}{Path tracer written in Rust using the
	\httplink[sceen-13]{https://github.com/attackgoat/screen-13} library as a Vulkan
	abstraction. It supports the Disney BSDF with Next Event Estimation.}
\cvline{VkJit}{Prototype Just In Time Compiler (JIT) with SPIRV/Vulkan as a
	backend, inspired by
	\httplink[Dr.Jit]{https://github.com/mitsuba-renderer/drjit}}
\cvline{Large Steps in Mitsuba3}{Implementation of the Large Steps in Inverse Rendering paper in Mitsuba3 using its PyTorch integration}
\cvline{Paper for EG2024}{
	Paper on real time neural rendering using multidimensional hash grid encodings.
	Based on NeRad, written in Mitsuba3.
	(Conditionally accepted)
}

\section{Programming Languages}
\cvline{C++}{Experience in modern C++ as well as C89 and C99. I have written
	Several projects in C/C++ from high level graphics applications to low level
	embedded software.}
\cvline{Python}{Experience using Python with PyTorch, Tensorflow and
	Mitsuba for ML.}
\cvline{Rust}{Experience using Rust for GPGPU and computer graphics. As
	Rust seems to be a promising new language for low and high level
	programming without some caveats of C++, I use it for my personal
	projects.}

\section{Other Abilities and Skills}
\subsection{Languages}
\cvlanguage{German}{native speaker}{}
\cvlanguage{English}{B2+/C1}{}



\end{document}
